% !TEX TS-program = xelatex
% !TEX encoding = UTF-8 Unicode
% !Mode:: "TeX:UTF-8"

\documentclass{resume}
\usepackage{zh_CN-Adobefonts_external} % Simplified Chinese Support using external fonts (./fonts/zh_CN-Adobe/)
% \usepackage{NotoSansSC_external}
% \usepackage{NotoSerifCJKsc_external}
% \usepackage{zh_CN-Adobefonts_internal} % Simplified Chinese Support using system fonts
\usepackage{linespacing_fix} % disable extra space before next section
\usepackage{cite}

\begin{document}
\pagenumbering{gobble} % suppress displaying page number

\name{李成}

\basicInfo{
  \email{im.lechain@gmail.com} \textperiodcentered\
  \phone{(+86) 155 7470 7978 } \textperiodcentered\
  \linkedin[李成]{https://www.linkedin.com/in/matrikslee}} \textperiodcentered\
 
\section{\faGraduationCap\ 教育背景}
\datedsubsection{\textbf{中国海洋大学}, 青岛}{2014/9 -- 2018/6}
\textit{本科}—工程学院—自动化专业

\hspace*{\fill}
\section{\faUsers\ 工作经历/项目经历}

\datedsubsection{\textbf{联发软件设计(深圳)有限公司}}{2019年4月 -- 至今}
\role{工程师}{TV芯片的HDMI驱动开发和维护}
\begin{onehalfspacing}
2019年开始——Mediatek 8K高端TV芯片的HDMI驱动开发与维护
\begin{itemize}
  \item Arm64平台的内存、寄存器和中断模型认知,linux kernel编码经验
  \item IC验证之后的公版驱动软件开发维护,处理内部公版测试中的驱动软件bug
  \item 客户项目的客户支持及BUG处理,并取得若干内部Award奖励
  \item HDMI规范:TMDS/FRL信号传输协议,CEC/I2C通信协议等
\end{itemize}
2021年开始——ARM64平台的Linux标准内核模块开发
\begin{itemize}
  \item arm64平台标准kernel模块的开发经验,
  \item 了解dma\_buf, mmap等kernel基础设施的用途及使用场景,并有一定的编码经验
  \item 内核模块代码在x86(x64)平台下的google test单元测试环境搭建,使用stub/mock方法
\end{itemize}
\end{onehalfspacing}
\hspace*{\fill}
\datedsubsection{\textbf{索尼精密部件(惠州)有限公司}}{2018年12月 -- 2019年3月}
\role{工程师}{生产线嵌入式设备及上位机软件开发与维护}
\begin{onehalfspacing}
1. 自行开发了用于生产线设备通信控制的上位机程序,用于调试与数据采集
\begin{itemize}
  \item 基于C\#和Win Form实现用户界面
  \item 基于事件的串口文本异步接收
  \item 实现收发文本的编码转换(ASCII,Unicode,UTF-8)
  \item 日志存储功能
\end{itemize}
2. 参与新产线的研发,主要负责数据收集和处理,并根据数据报告针对性调整设备参数
\end{onehalfspacing}

\hspace*{\fill}
\datedsubsection{\textbf{中国海洋大学——基于单片机的两轮自平衡小车} }{2015年10月 -- 2017年8月}
\role{单片机C语言开发}{参赛项目}
本项目为软硬件综合项目,队友负责硬件PCB设计等,我在团队中负责C代码的编写和维护,构建程序框架,并维护运行两年。

本项目的软件是单片机裸机程序,在单片机厂商提供的基础代码库上开发。
\begin{itemize}
  \item 传感器:陀螺仪、加速度计、电磁传感器、摄像头,
  \item 电机驱动:H桥驱动电路实现电机的正反转和速度控制,
  \item 控制算法:PID控制算法,对两轮平衡小车的姿态、速度以及方向的控制,
  \item 图像算法:二值图边界判定算法,寻找赛道边界,做为小车循迹的依据,
  \item 滤波算法:卡尔曼滤波器和平衡互补滤波器,对传感器采集数据滤波,过滤信号扰动。
\end{itemize}

% Reference Test
%\datedsubsection{\textbf{Paper Title\cite{zaharia2012resilient}}}{May. 2015}
%An xxx optimized for xxx\cite{verma2015large}
%\begin{itemize}
%  \item main contribution
%\end{itemize}

\hspace*{\fill}
\section{\faCogs\ 工作技能}
% increase linespacing [parsep=0.5ex]
编程语言:
\begin{itemize}[parsep=0.5ex]
  \item 精通C语言程序编写,
  \item 有一定的C++编码经验,
  \item 能编写简单的bash/makefile脚本,
  \item 能读写简单的汇编程序(AT\&T语法),
  \item 对其他多种编程语言(Java/Python/Rust/Lisp/C\#)有所了解。
\end{itemize}

通信协议:
\begin{itemize}
  \item 掌握I2C, RS232等简单的串行通信协议,
  \item 掌握HDMI规范的编码及通信协议,丰富的HDMI设备问题处理经验。
\end{itemize}

开发平台:
\begin{itemize}        
  \item 2年单片机裸机C语言程序开发经验,
  \item 2年的Arm64 Linux平台kernel驱动软件开发经验。
\end{itemize}

\hspace*{\fill}
\section{\faHeartO\ 获奖情况}
\datedline{第十二届全国大学生“恩智浦”杯智能汽车竞赛\quad\textit{全国二等奖,山东省一等奖}}{2017 年}
\datedline{“浪潮杯”山东省第七届 ACM 大学生程序设计竞赛\quad\textit{银牌}}{2016年}
\datedline{中国海洋大学第七届“朗讯杯”科技实训比赛\quad\textit{二等奖}}{2016年}
\datedline{中国海洋大学2015~2016学年\quad\textit{科技创新奖学金}}{2016年}
\datedline{全国大学生数学建模竞赛\quad\textit{山东赛区二等奖}}{2015年}

\hspace*{\fill}
\section{\faInfo\ 其他个人情况介绍}
% increase linespacing [parsep=0.5ex]
\begin{itemize}[parsep=0.5ex]
\item 数据结构和算法基础:高中参与信息奥林匹克竞赛,并获省级一等奖,大学参与大学生ACM竞赛,获省级二等奖
\item 传感器、信号处理、自动控制及控制理论基础:本科自动化专业课
\item 对计算机体系结构、操作系统及程序运行原理的理解和认识:完成对《深入理解计算机系统》(CS:APP)的自学,目前在读《程序员的自我修养:链接、装载与库》,《Linux内核完全注释》
\item 计算机网络基础:读完《TCP/IP详解》第一卷,并在Archlinux上有所实践
\item 热爱计算机技术,广泛涉猎一些计算机技术,学习能力强,个人PC使用Archlinux操作系统
\end{itemize}

%% Reference
%\newpage
%\bibliographystyle{IEEETran}
%\bibliography{mycite}
\end{document}
